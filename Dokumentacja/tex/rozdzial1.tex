	\newpage
\section{Ogólne określenie wymagań}		%1
%Ogólne określenie wymagań i zakresu programu (Czyli zleceniodawca określa wymagania programu) 




\subsection{Opis działania}  %1.1       

\hspace{0.60cm}Aplikacja na urządzenia mobilne umożliwiająca monitoring dokonań sportowych w dziedzinie biegania. Program ma umożliwić monitorowanie naszej aktywności biegowej. Aplikacja ma zapisywać przede wszystkim czas treningu, dystans, trasę uzyskaną dzięki modułowi GPS oraz intensywność treningu (np. wyliczając średnie tempo, średnią i maksymalną prędkość oraz skalone kalorie). Kożystając z aplikacji mamy mieć możliwość szczegółowej weryfikacji danych treningu, zarówno w trakcie jego trwania jak i po jego zakończeniu. Dodatkowo w podsumowaniu dzięki współpracy programu z GPS-em, można także spawdzić informacje o najniższym i najwyższym punkcie trasy. Szczegółowe statystyki mają pozwolić na analizę postępów i wyciągnięcie wniosków na przyszłość.

\hspace{0.60cm}Treningi mają być zapisywane w pamięci. Użytkownik ma mieć możliwość zobaczenia statystyk wybranego treningu.

\hspace{0.60cm}Aplikacja ma za zadanie także motywować nas do ćwiczeń, np. wysyłając nam powiadomienia, w ustalonym przez użytkownika momencie, o tym, że nie odbyliśmy jeszcze treningu.

\hspace{0.60cm}Poza pomiarami w trakcie treningu, aplikacja ma także liczyć kroki, kiedy działa w tle.






\subsection{Opis wyglądu}  %1.2


\hspace{0.60cm}Na głównej stronie treningu, którą widzi użytkownik po otwarciu aplikacji, Powinny znajdować się takie informacje jak:
\begin{itemize}
	
	\item czas trwania aktywności,
	\item prędkość w danym momencie, 
	\item średnia prędkość,
	\item dystans, 
	\item spalone kalorie, 
\end{itemize}

\hspace{0.60cm}Oprócz tego na stronie treningu Rys. \ref{rys:rysunek001} (s. \pageref{rys:rysunek001}) powinna znajdować się mapa, na której będzie pokazana aktualna pozycja uzytkownika, oraz przebyta trasa.

\hspace{0.60cm}Po zakończonym treningu aplikacja ma pokazać całą przebytą trasę na mapie oraz dać dostęp do szczegółowych statystyk treningu Rys. \ref{rys:rysunek002} (s. \pageref{rys:rysunek002}). Użytkownik ma mieć podgląd na wszystkie możliwe dane.

\begin{figure}[!htb]
	\centering
	\begin{minipage}{.5\textwidth}
		\centering
		\includegraphics[width=.4\linewidth]{rys/ekran_treningu.png}
		\caption{Ekran treningu}
		\label{rys:rysunek001}
	\end{minipage}%
	\begin{minipage}{.5\textwidth}
		\centering
		\includegraphics[width=.4\linewidth]{rys/ekran_podsumowania.png}
		\caption{Ekran podsumowania}
		\label{rys:rysunek002}
	\end{minipage}
\end{figure}

\hspace{0.60cm}Ekran krokomierza Rys. \ref{rys:rysunek003} (s. \pageref{rys:rysunek003}) ma zawierać tylko liczbę zrobionych w biezacym dniu kroków.

\begin{figure}[!htb]
	\centering
	\includegraphics[width=.2\linewidth]{rys/ekran_krokomierza.png}
	\caption{Ekran krokomierza}
	\label{rys:rysunek003}
\end{figure}


\hspace{0.60cm}Na dole aplikacji ma się znajdować menu za pomocą którego użytkownik może się przełączać pomiędzy ekranem krokomierza, treningu, odbytymi treningami i ustawieniami.

\hspace{0.60cm}Ekran zawierający historię odbytych treningów powinien przedstawiać je w postaci list. Ustawienia też powinny być przedstawione w postaci listy. 
 